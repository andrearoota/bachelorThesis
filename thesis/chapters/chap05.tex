\chapter{Conclusioni}

\section{Limitazioni dello studio}
Durante la progettazione, lo sviluppo e il successivo utilizzo del software, sono emerse diverse limitazioni che meritano un'attenta considerazione:

\begin{enumerate}
    \item \textbf{Restrizioni delle API}: le restrizioni imposte dalle API di Scopus, come il limite alle richieste e il numero massimo di chiavi, hanno limitato l'ampiezza e la profondità dell'analisi.

    \item \textbf{Limitazioni dell'$h$-index}: l'$h$-index, pur essendo un utile indicatore bibliometrico, ha le sue critiche e limitazioni \cite{ding2020exploring}. Non fornisce un quadro completo dell'impatto e dell'influenza di un autore ed è soggetto a variazioni a seconda della piattaforma bibliografica utilizzata e del fenomeno dell’autocitazionismo.

    \item \textbf{Aggiornamento dei dati}: la rapidità con cui si evolvono le pubblicazioni e le citazioni nel campo accademico e la possibilità di aggiornare e correggere le informazioni presenti sulla piattaforma Scopus rendono i dati raccolti rapidamente obsoleti influenzando la rilevanza e l'attualità delle analisi eseguite.
    
    \item \textbf{Analisi quantitativa e qualitativa}: la focalizzazione su analisi quantitative potrebbe tralasciare aspetti qualitativi importanti della ricerca, come l'influenza culturale, l'impatto sociale o l'importanza teorica, che non sono catturabili attraverso metodi bibliometrici.

\end{enumerate}

Queste sono le maggiori restrizioni riscontrate durante lo svolgimento del presente lavoro, portando a considerare evoluzione future differenti trattate nel prossimo paragrafo.

\section{Sviluppi futuri per il software}
Considerando le limitazioni evidenziate in precedenza, si propone un'evoluzione del progetto che differisce dall'attuale percorso. Per superare i vincoli imposti dalle API, si suggerisce di orientare l'analisi verso gruppi più specifici implementando funzionalità che consentano la selezione e l'analisi di autori associati a particolari università, campi di studio o stati, fornendo inoltre la possibilità di combinare questi filtri. Tale approccio potrebbe offrire analisi più mirate e dettagliate, contribuendo a una comprensione più profonda delle dinamiche accademiche specifiche.
Un'altra possibile evoluzione del progetto, sempre per mitigare i limiti imposti dalle API, includerebbe la funzionalità di permettere agli utenti di caricare un elenco specifico di autori da analizzare, attraverso l'upload di un file, o di selezionarli tramite una barra di ricerca. Questo miglioramento offrirebbe una maggiore personalizzazione nell'analisi, consentendo agli utenti di concentrarsi su gruppi di autori di loro interesse specifico.

Inoltre, le migliorie proposte offrirebbero la possibilità di creare rapidamente degli snapshot di specifici gruppi di ricerca, consentendo di osservare e analizzare la loro evoluzione nel tempo.
Per andare oltre le limitazioni dell'$h$-index, si potrebbe sviluppare un'analisi focalizzata sulle citazioni e sulla loro correlazione con autori appartenenti alla stessa università o alla stessa area geografica. Questo tipo di analisi permetterebbe di esplorare le dinamiche di collaborazione e di influenza all'interno di specifici ambienti accademici o comunità geografiche.

Infine, un'ulteriore evoluzione utile per semplificare e rendere più intuitiva l'usabilità del sistema potrebbe essere l'integrazione delle funzionalità di raccolta e aggregazione dati nell'interfaccia utilizzata per la creazione, gestione e visualizzazione delle analisi, eliminando la necessità di alternare tra la linea di comando e la dashboard.

\section{Commenti finali}
Questo lavoro ha raggiunto con successo gli obiettivi prefissati, stabilendo una base solida nel campo dell'analisi bibliometrica di gruppo attraverso lo sviluppo del presente software. L'applicazione non solo soddisfa le esigenze iniziali di acquisizione e analisi dei dati bibliometrici, ma introduce anche un'interfaccia utente intuitiva che semplifica l'accesso e l'esplorazione di queste analisi. Tuttavia, è chiaro che esistono ampi margini di manovra per il miglioramento poiché il potenziale per rendere il software e il flusso di lavoro più semplice è evidente e le possibilità di evoluzione sono molteplici. In definitiva, il lavoro svolto pone le basi per sviluppi futuri che possono rendere lo strumento ancora più prezioso per la comunità accademica e scientifica.